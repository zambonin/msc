\documentclass[12pt]{beamer}

\usetheme{Copenhagen}

\usepackage[english]{babel}
\usepackage[T1]{fontenc}
\usepackage[utf8]{inputenc}
\usepackage{enumitem}

\setitemize{
  itemsep=1em,
  label=\usebeamerfont*{itemize item}
    \usebeamercolor[fg]{itemize item}
    \usebeamertemplate{itemize item}
}

\definecolor{uottawa-garnet}{RGB}{143, 0, 26}
\usecolortheme[named=uottawa-garnet]{structure}

\setbeamertemplate{itemize items}[triangle]
\setbeamertemplate{navigation symbols}{}

\title{
  Security Analysis of the\\
  Rainbow-$\eta$ Signature Scheme
}

\author[G. Zambonin (\texttt{gustavo.zambonin@uottawa.ca})]
{\includegraphics[scale=0.15]{avatar} \\ \vspace{2mm}
  Gustavo Zambonin}
\institute{School of Electrical Engineering and Computer Science \\
  University of Ottawa
}
\date{March 4th, 2021 \\
  {\scriptsize International Research Mobility Symposium @ Carleton University}
}

\begin{document}

\begin{frame}[plain,noframenumbering]
  \titlepage{}
\end{frame}

\begin{frame}
  \frametitle{Context}
  \begin{itemize}
    \item Digital signatures are widely used to provide authenticity, integrity
        and non-repudiation to communications
    \item Their security is usually based on problems from number theory,
        solved efficiently by Shor's quantum algorithm
    \item \textbf{Quantum-safe cryptography} aims to create public-key
        cryptosystems secure against quantum computers
    \item We focus on the Rainbow signature scheme, based on systems of
        multivariate equations over finite fields
    \item Signature generation and verification are efficient but key
        sizes are large systems of equations (up to 100KB)
  \end{itemize}
\end{frame}

\begin{frame}
  \frametitle{Strategy}
  \begin{itemize}
    \item Large keys are a problem for embedded devices and efficient secure
        communications over a network (HTTPS)
    \item How can we \textbf{securely reduce} the key sizes of Rainbow without
        limiting parameter configuration?
    \item To generate a signature, random values are substituted into the
        private key, creating solvable systems of equations
    \item What if such values are pre-substituted into the private key? It may
        then be stored in a smaller fashion
    \item The result is Rainbow-$\eta$, which reduces the key pair by up to
        $71\%$, but lacks precise security analysis
  \end{itemize}
\end{frame}

\begin{frame}
  \frametitle{Analysis}
  \begin{itemize}
    \item Due to the pre-substitution of values in the private key of
        Rainbow-$\eta$, the randomness of signatures is not preserved
    \item A small number of signatures can be collected by an attacker and used
        to create an \textbf{equivalent private key}
    \item Complexity of the attack for usual parameters is non-trivial, but
        still much smaller than desired
    \item Proof-of-concept implementations of the attack in SageMath and Magma
        for small parameters
    \item Rainbow-$\eta$ is not recommended as a secure signature scheme
        (cf. \url{https://github.com/zambonin/msc})
  \end{itemize}
\end{frame}

\end{document}
