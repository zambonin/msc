\documentclass[draft, 12pt, a4paper, oneside]{memoir}
\usepackage[english]{babel}
\usepackage[T1]{fontenc}
\usepackage[utf8]{inputenc}

\begin{document}

% \chapter{Introduction}
% TODO write some pages here introducing the subject and problems
% TODO define hypothesis, objectives and scope before searching the literature

\section{Related work}\label{sec:related}

In this section, we make use of the previously well-defined boundaries with regards to the scope of our work in order to perform the necessary search in the literature. We recall that our main object of study are digital signature schemes which make use of Oil--Vinegar polynomials in their construction. More precisely, it is known that key sizes for instances of such schemes are currently impractical. Thus, we inspect works that aim to mitigate this situation, in order to identify common strategies and pitfalls.

We conduct our literature review by making use of the Google Scholar meta-indexer tool, which covers most scientific literature providers, and allows the user to efficiently delimit the desired scope. We identify three critical works in the history of signatures based on Oil--Vinegar polynomials~\cite{Patarin:199709,Kipnis:199904,Ding:200506} and conduct the search over works that directly cited at least one of those. We consider only publications written in English that additionally are not patents. From this set, we select relevant papers and present a summary below.

\subsection{Improvements to private key size}

Rainbow is evidently the most popular multivariate signature scheme based on Oil--Vinegar polynomials. However, schemes of that class that feature modifications with the intent of reducing key sizes have been proposed even before Rainbow itself was proposed. Chronologically, the first schemes with this feature are called stepwise triangular systems (STS) in their generalized form~\cite{Wolf:200603}. The inherent shape of their private keys, hence the name, is derived from the fact that there are restrictions in the choice of variables for each polynomial in the central map, on top of the Oil--Vinegar construction. This strategy indirectly enables a reduction in private key size. 

STS were found to be insecure in the same work that designed the general classification. The authors propose an inversion attack that recovers the original message from its corresponding signature, and additionally show that it is possible to build an equivalent private key. A similar idea introduced at roughly the same time are tame transformation schemes (TTS)~\cite{}, in which equations of the private key are required to present a minimum level of sparseness. This strategy remained unbroken for a longer duration than STS. 

In this period, STS and TTS were found to be specialized cases of the general Rainbow construction~\cite{Ding:200806}, due to their similarities in the usage of layers and the central map inversion procedure. These advances, as well as further cryptanalysis of Rainbow itself~\cite{Billet:200609} with the MinRank attack, led to the proposal of new parameters for TTS and Rainbow~\cite{Ding:200806}. This was achieved through the creation of the Rainbow band separation attack, an extension of the UOV reconciliation attack. Further modifications were suggested to salvage the previous schemes~\cite{}, but every such variant was broken through a key recovery attack~\cite{Thomae:201207} which stems from the generalization of the theory of equivalent keys.

% eq keys paragraph here

We note that in the case of STS and TTS, reducing the private key was not the main intention of the authors, and it is thus not trivial to estimate such gains through direct comparison to Rainbow. Indeed, the reduction of keys is a known strategy to increase the performance of operations on signatures. On the other hand, there are several proposals in the literature which focus on the modification of Rainbow with the sole purpose of reducing key sizes. This motivation partially stems from recent efforts to present quantum-safe cryptography as a valid alternative to traditional digital signature schemes~\cite{Bernstein:2008}.

A variant of Rainbow based on quadratic forms~\cite{Yasuda:201306} features a trade-off between private key and public key size in order to increase the performance of the signature generation step. It replaces the central map with a small invertible matrix, reducing the size of the private key by up to $91.2\%$ and increasing the public key threefold. However, it was found to be insecure directly afterwards~\cite{Hashimoto:201410} by algebraic cryptanalysis leading up to the discovery of equivalent keys.

NC-Rainbow~\cite{Yasuda:201202} is proposed as a novel strategy, based in non-commutative rings, used to reduce a private key by up to $75\%$. However, it was shown by independent researchers to be insecure through a combination of rank attacks and the UOV attack~\cite{Thomae:201209,Hashimoto:201302}. The variants MB-Rainbow~\cite{Yasuda:201305} and NT-Rainbow~\cite{Yasuda:201404} are based on the insertion of, respectively, sparse diagonal and non-triangular matrices into the central map, to reduce the private key by up to $40\%$. The authors merged MB- and NC-Rainbow into a single scheme MNT-Rainbow~\cite{Yasuda:201409}, shortening private keys by up to $76\%$. 

The parameters of MB- and NT-Rainbow were deemed vulnerable to the Rainbow band separation attack by the authors of~\cite{Peng:201706}. In this work, a new scheme called Circulant Rainbow is also proposed, which reduces the private key by up to $45\%$ due to the concept of rotating relations introduced to the central map. This approach and its analogous UOV variant~\cite{Peng:201803} were broken shortly after with an application of the UOV attack by the authors of~\cite{Hashimoto:201903}. The strategy of MB-Rainbow was applied to UOV in~\cite{Tan:201511}, alongside with the replacement of certain parts of the private key by values generated through linear recurring sequences. While these strategies reduce the private key by up to $89.1\%$, they were found to be insecure~\cite{Park:201803} with the discovery of an attack to efficiently find equivalent keys. 

In the ELSA variant~\cite{Shim:201712}, a hidden layer of quadratic equations is inserted to simplify the private key and thus achieve a reduction of up to $88\%$ in its size. Nonetheless, it was found that such structure introduced shortcuts enabling an attacker to find equivalent keys~\cite{Hashimoto:201909}. The SOV signature scheme~\cite{Shim:202001} is built by preventing overlap of quadratic terms in the central map and assigning a specific rank to the matrices associated with its polynomials, reducing private key size by up to $91.9\%$.

Some non intrusive approaches are also found in the literature. The authors of Lite-Rainbow-0~\cite{Shim:201512} propose the replacement of the entire private key with a small pseudorandom number generator (PRNG) seed. This shortens the private key by a factor of approximately $99.8\%$, but evidently increases the cost for signature generation. In~\cite{Borges:201209}, it is suggested to use the RC4 stream cipher in the private key generation of UOV~\cite{Borges:201209}, presenting analogous decreases in the private key size but no noticeable effects on signature generation performance. Such approach is also demonstrated to work with Rainbow~\cite{Dornelles:201910}. A similar result is also given in another work~\cite{Seo:201403}, in which AES-CTR is employed as the PRNG of choice.

The precomputation of UOV signatures such that the private key is not required is proposed in~\cite{Chen:201603}. This ``online-offline'' approach does not modify the key itself but can be used to yield signatures without the need to load the private key, thus reducing total storage requirements. Finally, the proposal by the authors of~\cite{Zambonin:201907} is discussed in more detail in~\cite{Bittencourt:201911} and further expanded in Section~\ref{}.

\subsection{Improvements to public key size}

In the case of modifications to the public key, there are mainly four approaches. The first strategy is based on the method of field lifting, where the resulting LUOV scheme~\cite{Beullens:201712} has the central map and public key lifted to an extension field, such that coefficients of those polynomial systems are now smaller. This is a trade-off between public key size and signature size, but it is general enough to be compatible with other public key reduction strategies. LUOV has been submitted to the standardization process by NIST~\cite{}, in which it was attacked with a new strategy based on differentials between the base and extension fields~\cite{Ding:201908}. An extension of the LUOV proposal to Rainbow~\cite{Duong:202003} was submitted before any attacks were known, but it is unclear if it is affected by the differential method.

A development by the name of block-anti-circulant (BAC) UOV~\cite{Szepieniec:201908} forces matrices representing the central map and public key to be BAC. This allows for a compression of both keys in the key pair, although only public key size improvements are featured on the aforementioned work. Still, the authors of~\cite{Furue:202004} show that the security levels of BACUOV are smaller than originally claimed through manipulation of the public key and the application of the UOV attack.

We also classify the work in~\cite{Szepieniec:201706} as relevant, since it is a general method motivated by the context of public key infrastructures, in which the size of a public key and signature bundle must be minimized. The strategy, which consists of a trade-off between signature and public key sizes with their summed length still reduced, was initially proposed for multivariate trapdoors, and subsequently generalized~\cite{Beullens:201808} to work in different security models and other signature schemes.

Finally, the ``cyclic'' approach initially proposed in~\cite{Petzoldt:201006} is, to the best of our knowledge, the most scrutinized method for public key reduction without compromises to the signature size. It explores the fact that specific parts of the public key do not contribute to the security of the scheme, and thus can be structured at will. A limitation of this approach is that one has no control over the generated private key, consequently preventing the combination of the cyclic approach with the methods above, for instance.

The CyclicRainbow variant~\cite{Petzoldt:201012} is an extension of the original method, which only discussed a modification of UOV. It is featured on the Rainbow submission for the second round of the standardization process conducted by NIST~\cite{Ding:201901}. It reduces public keys by up to $53.8\%$. Another strategy based on the aforementioned result is the usage of linear recurring sequences to generate such unimportant parts of the public key. It is summarized in two works~\cite{Petzoldt:201103,Petzoldt:201211} and reduce public keys by up to $56.6\%$. The mathematical relations that allow for this class of improvements are shown in Section~\ref{} for easier reading. We refer to the original work for more detailed explanations~\cite{Petzoldt:201307}.  This framework is also featured in some of the works previously described in Subsection~\ref{}.

\subsection{Improvements to total key pair size}

Additionally, there are works in the literature which manage to reduce both private and public key sizes through novel strategies. The authors of LWRS~\cite{Zhang:201208} create a scheme with performance-restricted devices in mind. The left affine transformation of the usual bipolar construction is dropped, and the ``minus'' method~\cite[Subsection 3.2.1]{Wolf:200511} is applied to the public key, of which the last $\delta$ equations are removed. Precisely $m (m + 1)$ field elements are removed in the private key, and a reduction factor of $1 - \frac{\delta}{m}$ is achieved on the public key. 

A modification that consists of adding cross-terms of oil variables in the last layer to the last two layers of Rainbow is proposed in~\cite{Tan:201603}. With this method, which is also applicable to UOV, the authors are able to reduce Rainbow private and public keys by up to $46.2\%$ and $36.8\%$, respectively.

The variant known as cubic UOV~\cite{Nie:201511} introduces a number of cubic polynomials in the central map of the private key. A consequence is that both keys of this variant are smaller when comparing to the original UOV scheme, although the focus of the original work is on reducing public keys. However, it was found to be insecure~\cite{Hashimoto:201712} through the recovery of equivalent keys. Repaired versions called CSSv and SVSv were proposed~\cite{Duong:201611}. The former removes most cubic polynomials and further unnecessary structure from the key pair, while the latter does not feature cubic polynomials at all and uses a sparse central map. These proposals achieve a further decrease in key sizes, being comparable to Rainbow in some instances. However, independent researchers found both strategies to be flawed~\cite{Shim:201711,Hashimoto:201712} to HighRank attacks and to key recovery attacks using equivalent keys.

% this appears to be the case outside our predefined scope
It is evident that there have been commendable efforts in the creation, validation and correction of signature schemes based on Oil--Vinegar polynomials. However, it should be noted that most works which introduce any kind of structure into the key pair eventually succumb to algebraic cryptanalysis. It is thus our intention with this review to suggest that such approaches should be applied with caution. 

\bibliographystyle{alpha}
\bibliography{ref}

\end{document}
