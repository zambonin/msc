\documentclass[draft, 12pt, a4paper, oneside]{memoir}
\usepackage[english]{babel}
\usepackage[T1]{fontenc}
\usepackage[utf8]{inputenc}
% \usepackage[a4paper,inner=2.5cm,outer=1.5cm,top=2.0cm,bottom=1.5cm,head=0.7cm,foot=0.7cm]{geometry}

\usepackage{amsmath, amsfonts, chronosys, enumitem}

\newcommand{\random}{\stackrel{\$}{\longleftarrow}}

\begin{document}

% \chapter{Introduction}
% TODO write some pages here introducing the subject and problems
% TODO define hypothesis, objectives and scope before searching the literature

\section{Related work}\label{sec:related}

In this section, we make use of the previously well-defined boundaries with regards to the scope of our work in order to perform the necessary search in the literature. We recall that our main object of study are digital signature schemes which make use of Oil--Vinegar polynomials in their construction. More precisely, it is known that key sizes for instances of such schemes are currently impractical. Thus, we inspect works that aim to mitigate this situation, in order to identify common strategies and pitfalls.

We perform a systematic review of the literature through the Google Scholar meta-indexer tool, which covers most scientific literature providers, and allows the user to efficiently delimit the wanted scope. We identify three critical works in the history of signatures based on Oil--Vinegar polynomials~\cite{Patarin:199709,Kipnis:199904,Ding:200506} and conduct the search over works that directly cited at least one of those. We consider only publications written in English, no older than the oldest of the aforementioned works, and that are not patents. From this set, we select relevant papers and present a summary below.

Rainbow is evidently the most popular multivariate signature scheme based on Oil--Vinegar polynomials. However, schemes of that class that feature modifications with the intent of reducing key sizes have been suggested even before Rainbow itself was created. Chronologically, the first schemes with this feature were called \emph{stepwise triangular systems} (STS) in their generalized form. The inherent shape of their private keys, hence the name, is derived from the fact that there are restrictions in the choice of variables for each polynomial in the central map system, on top of the Oil--Vinegar construction. This strategy indirectly enables a reduction in private key size. STS schemes were broken in the same work that designed the general classification~\cite{}.

A similar idea introduced at roughly the same time are \emph{tame transformation schemes} (TTS), in which equations in the private key are required to present a minimum level of sparseness. This strategy remained unbroken for a longer duration than STS. In this period, STS and TTS were found to be specialized cases of the general Rainbow construction after it was published~\cite{}, due to their similarities in the usage of layers and the central map inversion procedure. These advances, as well as the cryptanalysis of Rainbow itself~\cite{}, led to the development of new attacks that rendered TTS insecure~\cite{}. Further modifications were suggested to salvage and enhance the previous schemes, but every such variant was broken through a general attack due to Thomae--Wolf~\cite{}.

We note that in the case of STS and TTS, reducing the private key was not the main intention of the authors, and it is thus not trivial to estimate such gains through direct comparison to Rainbow. Indeed, the reduction of keys is a known strategy to increase the performance of operations on signatures, \emph{e.g.}~\cite{}. On the other hand, with the general description of the scheme available~\cite{}, there have been several works which focus on the modification of Rainbow with the sole purpose of reducing key sizes. This motivation partially stems from recent efforts to present quantum-safe cryptography as a valid alternative to traditional digital signature schemes~\cite{}. We discuss these ``newer generation''  works as follows, firstly addressing efforts towards reducing private keys.

NC-Rainbow was proposed in~\cite{}, with a novel strategy based in non-commutative
rings to reduce a private key by up to $75\%$. However, it was shown by independent researchers to be insecure~\cite{}. 

Other variants
called MB-Rainbow~\cite{Yasuda:201305:inproc} and
NT-Rainbow~\cite{Yasuda:201404:inproc} employ sparseness of maps to reduce the
number of terms in the private key by up to $40\%$.

The authors merged MB- and NC-Rainbow into a single scheme called
MNT-Rainbow~\cite{Yasuda:201409:article}, shortening private keys by up to
$76\%$. Nevertheless, the original schemes were deemed insecure and new
parameters were suggested in~\cite{Peng:201706:article}. It also proposes a new
scheme called Circulant Rainbow, which reduces the private key by up to $45\%$
due to the concept of rotating relations. Yet, it was broken shortly
after~\cite{Hashimoto:201810:misc}. Furthermore, it appears that the introduction
of structures in the private key is highly threatening to the overall security 
of a Rainbow scheme.



A scheme called Lite-Rainbow-0~\cite{} employs a small pseudorandom number generator (PRNG) seed to replace the private key entirely. This shortens the private key by a factor of approximately $99.8\%$, but greatly increases the cost for signature generation. % this is not structural, so it should go last


The authors of LWRS~\cite{} create a scheme with performance-restricted devices in mind. The left affine transformation of the usual bipolar construction is dropped, and the ``minus'' method~\cite[Subsection 3.2.1]{} is applied to the public key. The reduction factor is precisely $1 - \frac{\delta}{m}$ for the public key and $m (m + 1)$ for the private key. % this actually reduces both keys in the key pair

In the case of modifications to the public key, there are mainly two approaches.
The approach by the authors of~\cite{Petzoldt:201006:inproc} is, to the best of 
our knowledge, the main method for public key reduction without compromises to
the signature size. It explores the fact that specific parts of the public key
do not contribute to the security of the scheme. It is summarised in several
publications~\cite{Petzoldt:201012:inproc,Petzoldt:201103:inproc,Petzoldt:201211:inproc,Petzoldt:201307:phd}
and used in~\cite{Shim:201512:inproc} to construct Lite-Rainbow-1.

The authors of~\cite{Szepieniec:201706:inproc} lift the public and central maps
of a particular case of Rainbow to an extension field, reducing its public 
key size by an order of magnitude but increasing the signature size. This is 
compatible with the generalised scheme, as seen in~\cite{Beullens:201706:msc,Beullens:201712:inproc},
and can make use of the improvements to the public key given above. However,
both of these strategies cannot be combined with the private key improvements
previously cited. 

\emph{Hypothesis.} It follows from the analysis above that, to the best of our knowledge, no works in the literature attempt to reduce both private and public keys in Rainbow-like signature schemes. This is due to the intricate relationship between both keys, in which one is used to create the other. Therefore, our hypothesis is answered positively if a scheme description is uncovered that shortens both keys in the key pair.

\subsection{Research method}

We enumerate below a chronological list of activities that depict the process used to develop our research.

\begin{enumerate}[label=(\roman*), itemsep=1pt]
    \item Maintain a bibliography featuring works related to reduction of keys in Rainbow-like signature schemes;
    \item Classify strategies used by scheme variants to reduce private or public keys;
    \item Develop a strategy to modify keys that is independent from existing ones;
    \item Test the scheme derived from this strategy against known cryptanalytic methods, particularly ones that take advantage of concepts used to devise this variant;
    \item Measure the overall gains of the scheme with regards to key sizes, comparing the new variant with existing works;
    \item Provide a reference implementation to show that our proposal does not deviate from default inputs and outputs.
\end{enumerate}

\bibliographystyle{alpha}
\bibliography{ref}

\end{document}
