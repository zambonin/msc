\documentclass[openright]{report}
\usepackage[utf8]{inputenc}
\usepackage{chronosys, enumitem}
\usepackage[a5paper,inner=2.5cm,outer=1.5cm,top=2.0cm,bottom=1.5cm,head=0.7cm,foot=0.7cm]{geometry}

\title{Handling Vinegar Variables to Shorten Rainbow Key Pairs}
\author{Gustavo Zambonin}
\date{June 2019}

% \instituicao[a]{Universidade Federal de Santa Catarina}
% \departamento[o]{Departamento de Informática e Estatística}
% \curso[o]{Programa de Pós-Graduação em Ciência da Computação}
% \documento[o]{{Dissertação de Mestrado}}
% \titulo{Handling Vinegar Variables to Shorten Rainbow Key Pairs}
% \autor{Gustavo Zambonin}
% \grau{Mestre em Ciência da Computação}
% \local{Florianópolis}
% \data{12}{junho}{2019}
% \orientador[Orientador]{Prof.\ Ricardo Felipe Custódio, Dr.}

\begin{document}

\maketitle

\chapter{Introduction}

Handwritten signatures are one of the ubiquitous ways to ensure trust, in the case that a contract is sealed between parties. However, they present various logistical and security shortcomings. For instance, individuals are advised to be physically present to sign any documents, for they cannot be truthfully identified otherwise. Additionally, this kind of personal calligraphy can be forged with little effort. A solution coherent with the rise of the Information Age is found within mathematical frameworks known as signature schemes, that enable explicit assertions on the security of signatures.

A vibrant discipline of public-key cryptography, signature schemes are often associated with digital systems, in which transit of sensitive messages is expected to be secure. However, a specific scheme used within this situation drastically alters matters of security and performance compared to others. This is due to their natural connection to hard computational problems, that prevent forgery of signatures in various threat models. It is thus known that advances in computer engineering present direct consequences on the development of signature schemes.

Classical computers are electronic in nature, using circuits to perform computations. However, a new paradigm has emerged in the form of quantum computers, in which computations are performed outside the scope of classical mechanics. Although these computers are physically hard to implement, algorithms that make use of quantum phenomena have already been proposed and provably present a speedup compared to classical computers. One of these algorithms, due to Shor~\cite{}, simplifies the complexity of integer factorisation, and can be theoretically used to break widely used signature schemes.

Post-quantum cryptography addresses this issue by defining signature schemes which are either not known to be affected by quantum computers, or provably so~\cite{}. A popular technique employed to build quantum-resistant signatures is the usage of systems of equations over finite fields with multiple variables within the algorithms that rule the scheme. Primitives that use this strategy populate the area known as multivariate cryptography, and are extremely time- and space-efficient when creating and verifying signatures, while key generation is comparatively very slow and of large output.

Families of multivariate public-key cryptosystems are defined according to the types of finite fields used in the computations. This enables the focus on specific improvements, such as trade-offs between key and signature sizes, or larger security parameters. Particularly interesting are signature schemes based on the Oil--Vinegar principle, of which Rainbow~\cite{} receives primary attention. It performs computations on a single field and presents a balanced nature and simple, elegant definitions.

Large key sizes often impose constraints on the use of signature schemes in devices of limited storage and processing power. In multivariate cryptography, keys are often orders of magnitude larger than the ones used in conventional signature schemes. Due to this fact, several distinct approaches to reduce the key sizes of multivariate public-key cryptosystems have been proposed in the literature. In the case of Rainbow, to the best of our knowledge, these are distributed into methods that reduce exclusively private or public keys, but never both at the same time.

We address this fact by presenting a general framework that works for any Rainbow-like signature scheme, in which vinegar variables of the first layer are fixed in order to shorten the private key. This method allows for the use of techniques that reduce the public key on top of the existing modification, thus creating a Rainbow variant that has shorter private and public keys when compared to the classical scheme. We analyse the security implications of our proposal and show improvements with various parameter sets.

\section{Objectives}

We aim to create a novel method that enables the reduction of private and public key sizes in Rainbow-like signature schemes. Particularly, to support this goal, we wish to conduct a detailed study on the design of the Rainbow signature scheme and its variants. Further, the specific objectives are listed below:

\begin{enumerate}[label=(\roman*), itemsep=1pt]
  \item Classify strategies that enable compact representation of keys, \emph{e.g.} the use of particular data structures, algorithms or mathematical concepts;
  \item Establish the significance of structured private and public keys in the overall security of the schemes;
  \item Assess the consequences of shorter keys on the performance of the signature generation and verification steps;
  \item Produce a Rainbow-like signature scheme that can be combined with another chosen variant, allowing for the incorporation of benefits from both schemes.
\end{enumerate}

% \section{Contributions of this thesis}

\section{Chronology}

We outline the dates for special occasions that lead altogether to the conclusion of a postgraduate qualification, namely a Master of Science degree, by the author. In no specific order, these consist of assorted activities related to the underlying program:

\begin{enumerate}[label=(\roman*), itemsep=1pt]
  \item\label{enum:0} defense of the studies conducted over the qualification period;
  \item\label{enum:1} writing a master's thesis;
  \item\label{enum:2} the acceptance of a scientific paper by an academic conference or journal;
  \item\label{enum:3} a minimum of credits from a given curriculum;
  \item\label{enum:4} a qualification exam;
  \item\label{enum:5} an English proficiency exam;
  \item\label{enum:6} attendance in events promoted by the program. 
\end{enumerate}

\begin{figure}[htbp]
  \begin{chronology}[startyear=2018, stopyear=2021, height=3pt, dates=false, arrow=false, align=center]
    \chronoperiodecoloralternation{lightgray, gray}
    \definechronoevent{sp}[textstyle=\footnotesize, 
      datesseparation=/, conversionmonth=false, year=false]
    \chronoperiode[dateselevation=5pt]{2018}{2019}{}
    \chronoperiode[dates=false]{2019}{2020}{}
    \chronoperiode[dateselevation=5pt]{2020}{2021}{}
    \chronosp[markdepth=-25pt, datestyle=\textbf]{30/07/2018}{\textbf{Start of M.Sc.}}
    \chronosp[markdepth=-60pt]{15/02/2019}{Exam~\ref{enum:5}}
    \chronosp[markdepth=55pt]{09/03/2019}{1\textsuperscript{st} submission~\ref{enum:2}}
    \chronosp[markdepth=5pt]{02/05/2019}{1\textsuperscript{st} notification~\ref{enum:2}}
    \chronosp[markdepth=-25pt]{12/06/2019}{Exam~\ref{enum:4}}
    \chronosp[markdepth=20pt]{20/01/2020}{2\textsuperscript{nd} submission~\ref{enum:2}}
    \chronosp[markdepth=-60pt]{20/03/2020}{2\textsuperscript{nd} notification~\ref{enum:2}}
    \chronosp[markdepth=-25pt]{01/06/2020}{Manuscript~\ref{enum:1}}
    \chronosp[markdepth=55pt]{01/07/2020}{Defense~\ref{enum:0}}
    \chronosp[markdepth=5pt, datestyle=\textbf]{30/07/2020}{\textbf{End of M.Sc.}}
  \end{chronology}
  \caption{Timeline for relevant events throughout the M.Sc. studies of the author, over the course of four semesters.}
  \label{fig:1}
\end{figure}

Most of these events are shown in Figure~\ref{fig:1}. We highlight at least two paper submissions to conferences, but only the most relevant dates are presented to prevent visual pollution. Furthermore, items~\ref{enum:3} and~\ref{enum:6} are to be completed until the execution of item~\ref{enum:4}, and are not pictured due to the high distribution of dates related to these activities. We expect that items~\ref{enum:2} to~\ref{enum:6} are finished within the first year of the degree, and the remaining items on the second year.

\chapter{Methodological approach}

In this chapter, we describe the research strategy applied to produce this work. Recall that the object of study is the key sizes of Rainbow-like signature schemes. We define the main research question, parse the related body of literature, and finally apply several reproducible methods in order to infer conclusions. Indeed, this is essentially a description of the usual scientific method. In other words, we infer knowledge from mensurable experiments related to deductions, in the interest of solving specific limitations that come with the hypothesis.

\section{Literature review}

We inspect formal works such as journal and conference articles, and monographs. We restrict ourselves to works published exactly after the first formal description of Rainbow~\cite{}, and particularly select ones that deal with reduction of key sizes. To conduct this analysis, the Google Scholar web tool was employed. It is a meta-indexer that covers most scientific literature providers, and allows the user to input advanced parameters in order to correctly narrow the needed scope.

The exact query used is \texttt{rainbow AND signature AND crypto AND (scheme OR post-quantum OR cryptography)}, searched over publications in English, no older than 2005, and not including patents. It is important to note that the keywords are searched over the entirety of the paper, whenever available. The query aims to reduce the amount of noise generated by the common name of the scheme, and as such, returns $\approx 1{,}150$ results. 

This is close to the number of works obtained when grouping all analogous searches in other sites, such as the ACM Digital Library, IEEE Xplore, Scopus, Semantic Scholar and Springer Link. Indeed, Google Scholar additionally crawls other sources, \emph{e.g.} university and governmental repositories, and as such returns a slightly larger number of works. From this set, we select papers related to our object of study and present a summary below.

\subsubsection{Hypothesis.} It follows from the analysis above that, to the best of our knowledge, no works in the literature attempt to reduce both private and public keys in Rainbow-like signature schemes. 

\section{Research method}

\chapter{Theoretical framework}

\section{Multivariate cryptography}

\section{The Oil--Vinegar principle}

\section{Rainbow signature scheme}

\section{Strategies to reduce keys}

\chapter{Harnessing vinegar variables}

\section{Contributions}

\end{document}
