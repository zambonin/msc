\documentclass{article}
\usepackage[utf8]{inputenc}
\usepackage{chronosys}

\title{}
\author{}
\date{}

\begin{document}

\section{Chronology}

In this section, we outline the dates for special occasions that lead altogether to the conclusion of a postgraduate qualification, namely a Master of Science degree, by the author. These consist mainly of writing a master's thesis and the acceptance of a scientific paper by an academic conference or journal. Furthermore, assorted activities related to the underlying program are also needed, such as a minimum of credits from the program's curriculum, an English proficiency exam and attendance in events promoted by the program. 

\begin{figure}[htbp]
    \centering
    \begin{chronology}[startyear=2018, stopyear=2021, height=3pt, dates=false, arrow=false, align=center]
        \chronoperiodecoloralternation{lightgray, gray}
        \definechronoevent{sp}[textstyle=\footnotesize, 
            datesseparation=/, conversionmonth=false, year=false]
        \chronoperiode[dateselevation=5pt]{2018}{2019}{}
        \chronoperiode[dates=false]{2019}{2020}{}
        \chronoperiode[dateselevation=5pt]{2020}{2021}{}
        \chronosp[markdepth=-25pt]{30/07/2018}{Start of M.Sc.}
        \chronosp[markdepth=-60pt]{15/02/2019}{English proficiency exam}
        \chronosp[markdepth=50pt]{09/03/2019}{1\textsuperscript{st} submission}
        \chronosp[]{02/05/2019}{1\textsuperscript{st} notification}
        \chronosp[markdepth=-25pt]{12/06/2019}{Qualification exam}
        \chronosp[markdepth=50pt]{31/10/2019}{Summer school}
        \chronosp[]{20/01/2020}{2\textsuperscript{nd} submission}
        \chronosp[markdepth=-60pt]{20/03/2020}{2\textsuperscript{nd} notification}
        \chronosp[markdepth=-25pt]{01/06/2020}{Thesis manuscript}
        \chronosp[markdepth=50pt]{01/07/2020}{Defense}
        \chronosp[]{30/07/2020}{End of M.Sc.}
    \end{chronology}
    \caption{Timeline for relevant events throughout the M.Sc. studies of the author, over the course of four semesters.}
    \label{fig:1}
\end{figure}

\end{document}
