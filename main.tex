\documentclass[draft, 12pt, a4paper, oneside]{memoir}
\usepackage[english]{babel}
\usepackage[T1]{fontenc}
\usepackage[utf8]{inputenc}

\begin{document}

% \chapter{Introduction}
% TODO write some pages here introducing the subject and problems
% TODO define hypothesis, objectives and scope before searching the literature

\section{Related work}\label{sec:related}

In this section, we make use of the previously well-defined boundaries with regards to the scope of our work in order to perform the necessary search in the literature. We recall that our main object of study are digital signature schemes which make use of Oil--Vinegar polynomials in their construction. More precisely, it is known that key sizes for instances of such schemes are currently impractical. Thus, we inspect works that aim to mitigate this situation, in order to identify common strategies and pitfalls.

We conduct our literature review through the Google Scholar meta-indexer tool, which covers most scientific literature providers, and allows the user to efficiently delimit the wanted scope. We identify three critical works in the history of signatures based on Oil--Vinegar polynomials~\cite{Patarin:199709,Kipnis:199904,Ding:200506} and conduct the search over works that directly cited at least one of those. We consider only publications written in English that are not patents. From this set, we select relevant papers and present a summary below.

% TODO explain how they were broken
% if we're able to classify exactly which attack broke each scheme, then a table can be made

Rainbow is evidently the most popular multivariate signature scheme based on Oil--Vinegar polynomials. However, schemes of that class that feature modifications with the intent of reducing key sizes have been suggested even before Rainbow itself was proposed. Chronologically, the first schemes with this feature were called \emph{stepwise triangular systems} (STS) in their generalized form. The inherent shape of their private keys, hence the name, is derived from the fact that there are restrictions in the choice of variables for each polynomial in the central map system, on top of the Oil--Vinegar construction. This strategy indirectly enables a reduction in private key size. STS schemes were broken in the same work that designed the general classification~\cite{}.

A similar idea introduced at roughly the same time are \emph{tame transformation schemes} (TTS), in which equations in the private key are required to present a minimum level of sparseness. This strategy remained unbroken for a longer duration than STS. In this period, STS and TTS were found to be specialized cases of the general Rainbow construction after it was published~\cite{}, due to their similarities in the usage of layers and the central map inversion procedure. These advances, as well as the cryptanalysis of Rainbow itself~\cite{}, led to the development of new attacks that rendered TTS insecure~\cite{}. Further modifications were suggested to salvage and enhance the previous schemes, but every such variant was broken through a general attack due to Thomae--Wolf~\cite{}.

We note that in the case of STS and TTS, reducing the private key was not the main intention of the authors, and it is thus not trivial to estimate such gains through direct comparison to Rainbow. Indeed, the reduction of keys is a known strategy to increase the performance of operations on signatures, \emph{e.g.}~\cite{}. On the other hand, with the general description of the scheme available~\cite{}, there have been several works which focus on the modification of Rainbow with the sole purpose of reducing key sizes. This motivation partially stems from recent efforts to present quantum-safe cryptography as a valid alternative to traditional digital signature schemes~\cite{}. We discuss these ``newer generation''  works as follows, firstly addressing efforts towards reducing private keys.

NC-Rainbow was proposed in~\cite{}, with a novel strategy based in non-commutative rings to reduce a private key by up to $75\%$. However, it was shown by independent researchers to be insecure~\cite{}. Other variants called MB-Rainbow~\cite{} and NT-Rainbow~\cite{} employ sparseness of maps to reduce the number of terms in the private key by up to $40\%$. The authors merged MB- and NC-Rainbow into a single scheme called MNT-Rainbow~\cite{}, shortening private keys by up to $76\%$. Nevertheless, the original schemes were deemed insecure by the authors of~\cite{}. In this work, a new scheme called Circulant Rainbow is also proposed, which reduces the private key by up to $45\%$ due to the concept of rotating relations. This approach and its analogous UOV variant~\cite{} were broken shortly after~\cite{}.

The approach of MB-Rainbow was applied to UOV by the authors of~\cite{}, alongside with the replacement of certain parts of the private key by values generated through linear recurring sequences. While these strategies reduce the private key by up to $89.1\%$, they were found to be insecure~\cite{}. A variant of Rainbow based on quadratic forms~\cite{} features a trade-off between private key and public key size in order to increase the performance of the signature generation step. It replaces the central map with a small invertible matrix, reducing the size of the private key by up to $91.2\%$ and increasing the public key threefold. However, it was found to be insecure directly afterwards~\cite{}.

The ELSA variant~\cite{} uses a hidden layer of quadratic equations to simplify the private key and thus achieve a reduction of up to $88\%$ in its size. Nonetheless, it was subject to scrunity and broken shortly after~\cite{}. The SOV signature scheme~\cite{} is built by preventing overlap of quadratic terms in the central map and assigning a specific rank to the matrices associated with its polynomials, reducing private key size by up to $91.9\%$.

Some non intrusive approaches are also found in the literature. A scheme called Lite-Rainbow-0~\cite{} replaces the entire private key with a small pseudorandom number generator (PRNG) seed to. This shortens the private key by a factor of approximately $99.8\%$, but evidently increases the cost for signature generation. A similar approach uses the RC4 stream cipher to generate the private key of UOV instances~\cite{}, presenting a similar decrease in the private key size but no noticeable effects on signature generation performance. This approach is also demonstrated to work with Rainbow~\cite{}. A similar result is also given in another work~\cite{}, in which AES-CTR is employed as the PRNG of choice.

The precomputation of UOV signatures such that the private key is not required is proposed in~\cite{}. This ``online-offline'' approach does not modify the key itself but can be used to yield signatures without the need to load the private key, thus reducing total storage requirements.

Additionally, there are works in the literature which manage to reduce both private and public key sizes through novel strategies. The authors of LWRS~\cite{} create a scheme with performance-restricted devices in mind. The left affine transformation of the usual bipolar construction is dropped, and the ``minus'' method~\cite[Subsection 3.2.1]{} is applied to the public key, of which the last $\delta$ equations are removed. Precisely $m (m + 1)$ field elements are removed in the private key, and a reduction factor of $1 - \frac{\delta}{m}$ is achieved on the public key.

A modification that consists of adding cross-terms of oil variables in the last layer to the last two layers of Rainbow is proposed in~\cite{}. With this method, which is also applicable to UOV, the authors are able to reduce Rainbow private and public keys by up to $46.2\%$ and $36.8\%$, respectively.

The variant known as cubic UOV~\cite{} introduces a number of cubic polynomials in the central map of the private key. A consequence is that both keys of this variant are smaller when comparing to the original UOV scheme, although the focus of the original work is on reducing public keys. However, it was found to be insecure~\cite{}. Repaired versions called CSSv and SVSv were proposed~\cite{}. The former removes most cubic polynomials and further unnecessary structure from the key pair, while the latter strips

which feature another decrease in key sizes, with reductions of up to $\%$ for private keys and up to $\%$ for public keys.
% TODO cubic paragraph?


In the case of modifications to the public key, there are mainly two approaches.
The approach by the authors of~\cite{Petzoldt:201006:inproc} is, to the best of 
our knowledge, the main method for public key reduction without compromises to
the signature size. It explores the fact that specific parts of the public key
do not contribute to the security of the scheme. It is summarised in several
publications~\cite{Petzoldt:201012:inproc,Petzoldt:201103:inproc,Petzoldt:201211:inproc,Petzoldt:201307:phd}
and used in~\cite{Shim:201512:inproc} to construct Lite-Rainbow-1.

The authors of~\cite{Szepieniec:201706:inproc} lift the public and central maps
of a particular case of Rainbow to an extension field, reducing its public 
key size by an order of magnitude but increasing the signature size. This is 
compatible with the generalised scheme, as seen in~\cite{Beullens:201706:msc,Beullens:201712:inproc},
and can make use of the improvements to the public key given above. However,
both of these strategies cannot be combined with the private key improvements
previously cited. 

\emph{Hypothesis.} It follows from the analysis above that, to the best of our knowledge, no works in the literature attempt to reduce both private and public keys in Rainbow-like signature schemes. This is due to the intricate relationship between both keys, in which one is used to create the other. Therefore, our hypothesis is answered positively if a scheme description is uncovered that shortens both keys in the key pair.

\bibliographystyle{alpha}
\bibliography{ref}

\end{document}
